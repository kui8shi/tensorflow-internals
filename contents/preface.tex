\chapter{序文} 
\label{ch:preface}

\section*{本書の位置づけ}

\begin{content}

本書は、TensorFlowの内部動作原理を解析する書籍です。TensorFlowを使用して機械学習モデルを構築する方法や、TensorFlowの最適な使用方法については説明しません。本書では、TensorFlowのソースコードを解析することで、TensorFlowのシステムアーキテクチャ、ドメインモデル、動作原理、および実装パターンなどの関連内容を明らかにし、その内部の知識を解明します。

\end{content}


\section*{対象読者}

\begin{content}

本書は、読者が機械学習の関連する基本概念と理論を理解し、機械学習の基本的な方法論を知っていることを前提としています。同時に、読者がPython、C++などのプログラミング言語に精通していることも想定しています。

本書は、TensorFlowの内部設計を深く理解したい、TensorFlowのシステム設計とパフォーマンス最適化を改善したい、そしてTensorFlowの主要技術の設計と実装を探求したいシステムアーキテクト、AIアルゴリズムエンジニア、AIソフトウェアエンジニアに適しています。

\end{content}

\section*{読み方}

\begin{content}

本書を初めて読む場合は、順を追って読むことをお勧めします。上級ユーザーの場合は、興味のある章を選んで読むこともできます。TensorFlowを初めて使用する場合は、ソースコードから完全にTensorFlowをビルドすることをお勧めします。これにより、システムの構築方法と基本的なコンポーネントライブラリの依存関係を理解することができます。

また、TensorFlowを使用して具体的なアプリケーションを実践することをお勧めします。これにより、TensorFlowのシステム動作に対する認識と理解が深まり、一般的なAPIの使用方法と動作原理に慣れることができます。本書を読む際には、TensorFlowの重要なコードを同時に読むことを強くお勧めします。コードを読む最良の実践については、本書の付録Aをご覧ください。

\end{content}

\section*{バージョンに関する説明}

\begin{content}

本書の執筆時点で、TensorFlowの安定リリースバージョンは1.2です。本書で説明しているAPIの一部が将来廃止される可能性や、システムの実装の一部が将来のバージョンで変更または削除される可能性があることをご了承ください。

同時に、問題の本質をより直接的に説明するために、本書の一部のコードに局所的な再構築を行っています。一部の例外処理分岐やログ出力、さらには一部のオプションパラメータリストを削除しています。しかし、このような局所的な再構築は、読者がシステムの主要な動作特性を理解する上で影響を与えず、むしろシステムの動作原理を把握するのに役立ちます。

また、計算グラフの表現を簡略化するために、本書の計算グラフはTensorBoardからのものではなく、簡略化された等価なグラフ構造を採用しています。同様に、簡略化されたグラフ構造も、読者の実際のグラフ構造に対する認識と理解を低下させることはありません。

\end{content}

\section*{オンラインサポート}

\begin{content}

読者とより良くコミュニケーションを取るために、Githubに正誤表および関連する補足説明を設置しました。個人の経験と能力には限りがあり、限られた時間内では間違いを犯すことは避けられません。読者の皆様が読書の過程で関連する誤りを発見された場合は、Pull Requestを提出していただけると幸いです。これにより、他の人々が同じ落とし穴に陥るのを防ぎ、知識の共有をより円滑に、より容易にすることができます。深く感謝いたします。

同時に、私の簡書(Jianshu)をフォローしていただければ幸いです。関連する記事を継続的に更新し、より多くの友人と一緒に学び、進歩していきたいと思います。

\begin{enum}
  \eitem{\ascii{Github: \script{\url{https://github.com/horance-liu/tensorflow-internals-errors}}}}
  \eitem{\ascii{簡書:\script{\url{http://www.jianshu.com/u/49d1f3b7049e}}}}
\end{enum}

\end{content}

\section*{謝辞}

\begin{content}

仕事の合間を縫って本書の校閲を行い、多くの修正意見を提供してくれた妻の劉梅紅に感謝します。

\end{content}
