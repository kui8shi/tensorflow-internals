\chapter{継続的学習} 
\label{ch:learning}


\section{文字の解説}

\begin{content}

\begin{remark}
読書には三つの到達点がある。心に到達し、目に到達し、口に到達することである。- 朱熹『訓学斎規』
\end{remark}

私が生まれたとき、父は私に\emph{劉光雲}という名前をつけ、\quo{光}の世代を継ぎ、単字で\quo{雲}とした。しかし、学校に入ってからは、その意味がわからなくなった。ある日、国語の先生が文字を解説して言った:\quo{聡とは、耳に到達し、目に到達し、口に到達し、心に到達することである}。遅ればせながら出会えた感覚で、私は\quo{聡}の字に惹かれ、自分の名前を\emph{劉光聡}に変えた。

面白いことに、それ以来、母は私の学習を心配することはなくなった。

\subsection{選択}

耳に到達し、その善きものを選んで従い、善くないものを選んで改める。巨大な関数や大きな論理を好む人もいて、なぜそうするのかと問われれば、効\ascii{(hai)}率\ascii{(zi)}のためだと美化する。一方、私は階層感のあるコードスタイル、簡潔で意図が明確なものを好む。

他人の経験は確かに重要だが、私たちは選択的に受け入れる必要があり、ただ聞くだけではいけない。大家に惑わされてはいけない、大家も時には間違えることがある。重要なのは自分で考え、賢明に判断することだ。特にこの浮ついた世の中では、地面を走れるものは皆自分を\quo{大家}と呼んでいる。

\subsection{抽象化}

目に到達し、外物を取り除き、本質を直接探る。一目で見えるものは全て仮象であり、見えないもの、触れられないものこそが往々にして本質である。平坦で直接的な論理を好む人もいるが、私はより抽象化を好み、本質を明らかにするプロセスを楽しみとしている。

抽象化には確かに複雑さが存在する。しかし、この複雑さには文脈がある。みんなが似たような経験を持っていれば、抽象化は自然とパターンになる。それは美であり、コミュニケーションの媒体である。

相手が文脈を欠いていれば、抽象化は当然難しい。いわゆる簡単さとは、問題の本質を明らかにし、そのために最小の代価を払うこと。平坦で直接的なものではなく、簡単さとは門外漢には決して感じ取れない美的感覚なのだ。

度を超すと及ばず、盲目的な抽象化は必然的に不必要な複雑さを増す。大規模な事前設計や、顧客の様々な要求を語り、ソフトウェア設計における様々な変化を語り、盲目的に抽象化するようなものだ。

\subsection{共有}

口に到達し、道を伝え、業を授け、疑問を解く。共有は生活の信念であり、共有を理解すると同時に、自然と存在の意義を理解する。私は自分の知識を共有するのが好きで、それを学習の動機として、問題の本質を徹底的に理解するよう自分を促す。

共有できるからこそ、知識は自然と自分のものになる。毎日の\ascii{Code Review}で、私はチームメンバーに積極的に共有するよう奨励している。一つは差のないチームを促進するため、もう一つは共有者が問題の本質を徹底的に理解するのを助けるためだ。

他人に自分の観点を信じてもらうには、信じる理由を与えることが重要だ。共有することで、自分の表現力を鍛える助けになる。これには長期的な\emph{意図的な練習}が必要だ。

\subsection{理解}

心に到達し、学んで考え、考えれば得られ、考えなければ得られない。自分で独立して考え、帰納し総括した知識こそが、真に自分のものとなる。

私は知識を総括するのに図表を使うのが好きだ。一つには図の表現力が文字よりはるかに大きいこと。もう一つは、図を描くことで自分に問題の本質を徹底的に理解することを強いるからだ。

\end{content}

\section{成長の道}

\begin{content}

\subsection{重複の排除}

コードは重複を排除する必要があり、仕事の習慣も重複を排除する必要がある。固定された仕事の状態に拘らないこと。重複する仕事の状態は往々にして人を快適さの錯覚に陥らせ、\emph{3年効果}の危機に陥らせる。

\subsection{知識の抽出}

まず、私たちが学ぶのは情報ではなく、知識である。知識には価値があるが、情報には価値がない。自分でスクリーニングし、抽出し、まとめてこそ、情報を知識に変えることができる。

\subsection{習慣になる}

知識は忘れやすい。知識を行動に移し、自分の仕事の状態に融合させてこそ、永久的に自分の財産となる。

例えば、ショートカットキーの使用は、意図的に記憶しようとするのではなく、自分の仕事の習慣の一つになるべきだ。繰り返しの労働はせず、\ascii{Shell}を使って自動化の程度を提供し、\ascii{Shell}を仕事の効率を上げる道具とし、それが仕事の習慣となるようにする。

\subsection{知識の更新}

私たちは既存の知識体系を常に更新する必要がある。特に私たちは知識爆発の時代にいる。私は教条を信奉する信者を嫌う。簡単な例を挙げると、古いコード規範では\code{if (NULL != p)}のような\code{YODA Notation}の慣用句を要求することがよくある。しかし、このような表現はコンパイラには喜ばれるが、プログラマーには非常に不親切だ。

\ascii{\quo{if you are at least 18 years old}}は明らかに\ascii{\quo{if 18 years is less than or equal to your age}}より英語の表現習慣に合っている。

この慣用句を擁護する人もいるが、現代のコンパイラはこの種の誤用に通常警告を報告する。また、\ascii{TDD}開発のリズムを保ち、小さなステップで前進すれば、このような低レベルのエラーはテストの網をくぐり抜けるのは難しい。

\subsection{自己のリファクタリング}

学んでから不足を知り、教えてから困難を知る。原点に留まらず、常に自分の知識体系をリファクタリングするべきだ。

\ascii{OO}設計に入門したばかりの頃、私はあらゆるところでデザインパターンを使っていた。私が見たすべての本が、デザインパターンがいかに素晴らしいかを語っていたからだ。後に進化的設計、シンプルデザイン、過剰設計についての見解を見て、理性に立ち返ることができた。

\subsection{専門技術の追求}

人のエネルギーには限りがあり、一人で世界のすべての知識を習得することはできない。プログラミング言語の選択に迷うよりも、方法論の内在する本質を徹底的に理解する方がよい。多くのフレームワークで決断できないよりも、実際に努力し、問題自体に焦点を当てる方がよい。

要するに、博学だが精通していないことは警戒すべきだ。

\end{content}
