\begin{savequote}[45mm]
\ascii{Any fool can write code that a computer can understand. Good programmers write code that humans can understand.}
\qauthor{\ascii{- Martin Flower}}
\end{savequote}

\chapter{Saver} 
\label{ch:saver}

\section{Saver}

\begin{content}

深層学習モデルの学習過程では、モデルの状態を保存することが重要です。これにより、学習の中断や再開が可能になり、\tf{}では、この機能をチェックポイント(\ascii{Checkpoint})と呼びます。

\code{Saver}は学習モデルの状態を保存および復元するためのクラスです。これにより、学習の途中経過や最終的な結果を保存し、必要に応じて以前の状態に戻すことができます。また、学習モデルの移植や共有も可能になります。主に、\code{Saver}は以下の2つの機能を提供します:

\begin{enum}
  \eitem{\code{save}: 学習モデルの現在の状態を指定されたチェックポイントファイルに保存します;}
  \eitem{\code{restore}: チェックポイントファイルから学習モデルの状態を復元します} 
\end{enum}

\subsection{使用方法}

例えば、以下のようなシンプルな例を考えてみましょう。まず、グラフを構築し、それを実行してチェックポイントに保存します。

\begin{leftbar}
\begin{python}
# construct graph
v1 = tf.Variable([0], name='v1')
v2 = tf.Variable([0], name='v2')

# run graph
with tf.Session() as sess:
  sess.run(tf.global_variables_initializer())
  saver = tf.train.Saver()
  saver.save(sess, 'ckp')
\end{python}
\end{leftbar}

次に、保存されたチェックポイントからモデルを復元します。

\begin{leftbar}
\begin{python}
with tf.Session() as sess:
  saver = tf.import_meta_graph('ckp.meta')
  saver.restore(sess, 'ckp')
\end{python}
\end{leftbar}

\subsection{チェックポイントの構造}

典型的な\code{Saver.save}操作後、以下のファイルが生成されます:

\begin{leftbar}
\begin{python}
├── checkpoint
├── ckp.data-00000-of-00001
├── ckp.index
├── ckp.meta
\end{python}
\end{leftbar}

\subsubsection{インデックスファイル}

インデックス(\ascii{index})ファイルは基本的にハッシュテーブル(\code{tensorflow::table::Table})の形式で、各\ascii{Tensor}の名前をキーとし、その\ascii{Tensor}のメタデータ情報を値として持ちます。各\ascii{Tensor}はそれぞれ1つのデータ(\ascii{data})ファイルに保存され、このデータファイルの位置も記録されており、素早くアクセスできるようになっています。

\subsubsection{データファイル}

データ(\ascii{data})ファイルには実際の変数\ascii{(Variable)}の値が含まれています。\code{restore}時に変数を復元する際、まずインデックスファイルから各変数の対応するデータファイルを見つけ、そこからインデックスに基づいて変数の値を読み取り、変数の値を復元します。

\subsubsection{メタファイル}

メタファイル(\ascii{meta})には\code{MetaGraphDef}のシリアル化されたデータが含まれており、\code{GraphDef, SaverDef}などのメタデータを含んでいます。

このファイルには変数の値以外のモデルのメタデータが保存されており、計算グラフの構造やその他のモデル関連の情報が含まれています。例えば、復元(\ascii{Restore})時には、まず\code{tf.import\_meta\_graph}を使って\code{GraphDef}を復元し、次に\code{SaverDef}を復元します。これにより、元の計算グラフの\code{Graph}オブジェクトが復元され、変数の値を復元するための\code{Saver}オブジェクトも作成され、最後に\code{Saver.restore}を使って各変数の値を復元します。

注意すべき点として、\code{Saver.restore}を呼び出す前に、必ず\code{tf.import\_meta\_graph}を呼び出す必要があります。そうしないと、変数の定義が存在せず、データを正しく復元できません。

\subsubsection{チェックポイントファイル}

\ascii{Checkpoint}ファイルは最新のチェックポイントファイル(\ascii{Checkpoint File})のリストを含んでおり、このリストを使って最新のインデックスやデータファイルを見つけることができます。\code{tf.train.latest\_checkpoint}を使用すると、最新のチェックポイントファイルを見つけることができます。

また、\ascii{Checkpoint}ファイルは基本的に一連のチェックポイントファイルのセットで、これらのファイルセットは学習の進行に伴って更新されます。学習プロセス中に定期的に保存することで、チェックポイントは増えていきますが、ディスク容量を節約するために古いチェックポイントを削除することもあります。これを制御するために、以下の2つのパラメータがあります:

\begin{enum}
  \eitem{\code{max\_to\_keep}: 保持する最新のチェックポイントの最大数。新しいチェックポイントが作成されると、このチェックポイント数が\code{max\_to\_keep}を超える場合、最も古いチェックポイントが削除されます。デフォルト値は\ascii{5}です;}
  \eitem{\code{keep\_checkpoint\_every\_n\_hours}: 学習が一定時間(\code{n}時間)続いた後に1つのチェックポイントを保存し、これは永久に保持されます。デフォルトでは無効になっています。} 
\end{enum}

このように\ascii{Checkpoint}ファイルは基本的にチェックポイントファイルのセットで、これらのファイルセットは学習の進行に伴って更新されます。この仕組みにより、最新の状態を保持しながら、必要に応じて過去の状態に戻ることができます。

\subsection{実装}

\subsubsection{シリアル化の実装}

モデルのシリアル化のために、\code{Saver}は内部で、グラフ内に\code{SaveV2}という特別な\ascii{OP}を作成します。この中で、\code{file\_name}は1つの\ascii{Const}の\ascii{OP}で、チェックポイントファイルの名前を表します。\code{tensor\_names}も1つの\ascii{Const}の\ascii{OP}で、保存する学習モデルの\ascii{Tensor}名のリストを表します。

\begin{figure}[!htbp]
\centering
\includegraphics[width=0.5\textwidth]{figures/py-saver-save-model.png}
\caption{Saver:シリアル化の実装}
 \label{fig:py-saver-save-model}
\end{figure}

\subsubsection{復元の実装}

同様に、復元プロセスでは、\code{Saver}は内部で、各学習モデルに対して1つの\code{RestoreV2}という特別な\ascii{OP}を作成します。この中で、チェックポイントファイルから復元される変数のデフォルト値(Initializer)も作成され、これは1つの\code{Assign}の\ascii{OP}になります。

ここでも、\code{file\_name}は1つの\ascii{Const}の\ascii{OP}で、チェックポイントファイルの名前を表します。\code{tensor\_names}も1つの\ascii{Const}の\ascii{OP}で、保存する学習モデルの\ascii{Tensor}名のリストを表し、その数は1です。

\begin{figure}[!htbp]
\centering
\includegraphics[width=0.9\textwidth]{figures/py-saver-restore-model.png}
\caption{Saver:復元の実装}
 \label{fig:py-saver-restore-model}
\end{figure}

\end{content}
