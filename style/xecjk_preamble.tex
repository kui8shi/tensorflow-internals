%%%%%%%%------------------------------------------------------------------------
%%%% 日本語用xeCJKパッケージ設定

\usepackage{xltxtra,fontspec,xunicode}

%% \CJKsetecglue{\hskip 0.15em plus 0.05em minus 0.05em}
%% slantfont: 斜体を許可
%% boldfont: 太字を許可
%% CJKnormalspaces: 日本語文字間のスペースを無視し、日本語と英語の間のスペースは保持。
%% CJKchecksingle: 単一の日本語文字が行を占めるのを避ける。
\usepackage[slantfont, boldfont]{xeCJK} 

%% 日本語の改行を行う
\XeTeXlinebreaklocale "ja"             

%% TeXの改行に一定の自由度を与える
\XeTeXlinebreakskip = 0pt plus 1pt minus 0.1pt

%%%% xeCJK設定終了
%%%%%%%%------------------------------------------------------------------------

%%%%%%%%------------------------------------------------------------------------
%%%% 日本語フォント設定

%% 句読点スタイルを設定
\punctstyle{plain}                                        
                                                     
%% デフォルトの日本語フォントを設定
\setCJKmainfont[BoldFont={Noto Sans CJK JP}, ItalicFont={Noto Serif CJK JP}]{Noto Serif CJK JP}
%% 日本語のサンセリフフォントを設定
\setCJKsansfont[BoldFont={Noto Sans CJK JP}, ItalicFont={Noto Serif CJK JP}]{Noto Sans CJK JP}
%% 等幅フォントを設定
\setCJKmonofont{Noto Sans Mono}

%% 英語のセリフフォント
\setmainfont{Lucida Bright}
%% 英語の等幅フォント
\setmonofont{Monaco}
%% 英語のサンセリフフォント
\setsansfont{Optima}

%% 新しいフォントを定義
\setCJKfamilyfont{mincho}{Noto Serif CJK JP}
\setCJKfamilyfont{gothic}{Noto Sans CJK JP}
\setCJKfamilyfont{maru}{M+ 1c}

%%カスタム英語フォント
\newfontfamily\couriernew{Lucida Grande}
\newfontfamily\optima{Optima}
\newfontfamily\lucida{Lucida Bright}

\newcommand{\ascii}[1]{{\sffamily #1}}
\newcommand{\speak}[1]{{\itshape #1}}
\renewcommand{\emph}[1]{{\gothic #1}}

%% カスタム明朝体
\newcommand{\mincho}{\CJKfamily{mincho}}
%% カスタムゴシック体
\newcommand{\gothic}{\CJKfamily{gothic}}
%% カスタム丸ゴシック体
\newcommand{\maru}{\CJKfamily{maru}}

%%%% 日本語フォント設定終了
%%%%%%%%------------------------------------------------------------------------

%%%%%%%%------------------------------------------------------------------------
%%%% 日本語文書に関するいくつかの再定義

%% 数学公式定理の再定義

\newtheorem{example}{例}[section]                                   
\newtheorem{algorithm}{アルゴリズム}
%% セクション別に番号付け
\newtheorem{theorem}{定理}[section]                         
\newtheorem{definition}{定義}
\newtheorem{axiom}{公理}
\newtheorem{property}{性質}
\newtheorem{proposition}{命題}
\newtheorem{lemma}{補題}
\newtheorem{corollary}{系}
\newtheorem{condition}{条件}
\newtheorem{conclusion}{結論}
\newtheorem{assumption}{仮定}

\newtheorem{principle}{原則}[section]
\newtheorem{regulation}{規則}[section]
\newtheorem{advise}{提案}[section]
\newtheorem{concept}{概念}[section]

\usepackage{titlesec}

\renewcommand{\partname}{}
\renewcommand{\thepart}{第\Roman{part}部}

%% 章節などの名前を再定義
\renewcommand{\contentsname}{目次}
\renewcommand{\abstractname}{要旨}
\renewcommand{\indexname}{索引}
\renewcommand{\listfigurename}{図目次}
\renewcommand{\listtablename}{表目次}
\renewcommand{\figurename}{図}
\renewcommand{\tablename}{表}
\renewcommand{\appendixname}{付録}
\renewcommand{\appendixpagename}{付録}
\renewcommand{\appendixtocname}{付録}
\renewcommand\refname{参考文献}

%%本文環境を設定
\newenvironment{content}{%
  \setlength{\parskip}{0.5\baselineskip}
  \begin{spacing}{1.5}
}{%
  \end{spacing}
  \setlength{\parskip}{-0.5\baselineskip}
  \vskip -0.5\baselineskip
}

% 短い物語を挿入する構文:
% \begin{story}
%   \begin{center}
%     \inlinetitle{タイトル}
%   \end{center}
% \end{story}
\newenvironment{story}
{
  \setlength{\parskip}{0.5\baselineskip}
  \hbox to \textwidth{\hfil\rule{\linewidth}{0.5mm}\hfil}
  \begin{spacing}{1.5}
}{%
  \end{spacing}
  \hbox to \textwidth{\hfil\rule{\linewidth}{0.5mm}\hfil}
  \setlength{\parskip}{-0.5\baselineskip}
  \vskip -0.5\baselineskip
}

%% chapter、section、subsectionのフォーマットを設定
\titleformat{\chapter}[display]{\flushright\huge\gothic}{\thechapter}{1em}{\textbf}
\titleformat{\section}[block]{\flushleft\Large\gothic}{\thesection}{1em}{\textbf}
\titleformat{\subsection}{\large\gothic}{\thesubsection}{0.5em}{\textbf}
\titleformat{\subsubsection}{\normalsize\gothic}{\thesubsubsection}{0.5em}{\textbf}

\titlespacing{\chapter}{0pt}{50pt}{40pt}
\titlespacing{\section}{0pt}{20pt}{10pt}
\titlespacing{\subsection}{0pt}{15pt}{10pt}

%% 章のフォーマットを設定
\usepackage{quotchap}

\renewcommand\chapterheadstartvskip{
   \vspace*{-5\baselineskip}
}

\renewcommand\chapterheadendvskip{
   \vspace*{0.5\baselineskip}
}

\usepackage{helvet}
\renewcommand\sectfont{\gothic\bfseries}

\newcommand\refig[1]{{\itshape \figurename\ascii{\ref{fig:#1}(\pageref{fig:#1}ページ)}}}
\newcommand\reftbl[1]{{\itshape \tablename\ascii{\ref{tbl:#1}(\pageref{tbl:#1}ページ)}}}

\renewcommand{\footnoterule}{\vspace*{3pt}%
  \hrule width 0.382\textwidth height 0.4pt \vspace*{2.6pt}}

% 備考
\newenvironment{remark}{\par\vskip10pt\footnotesize\itshape} % 備考の上部の垂直方向の空白とより小さいフォントサイズ
\begin{list}{}{
\leftmargin=35pt % 左側のインデント
\rightmargin=25pt}\item\ignorespaces % 右側のインデント
\makebox[-2.5pt]{\begin{tikzpicture}[overlay]
\node[draw=red!60,line width=1pt,circle,fill=red!25,font=\sffamily\bfseries,inner sep=2pt,outer sep=0pt] at (-15pt,0pt){\textcolor{red}{R}};\end{tikzpicture}}
\advance\baselineskip -1pt}{\end{list}\vskip5pt}

%%%% 日本語設定終了
%%%%%%%%------------------------------------------------------------------------
