%%%% listingsパッケージのセットアップ(ソースコードの貼り付け用)
%% ソースコードの貼り付けと部分的なコードハイライト機能
\usepackage{listings}
\usepackage{color}
\DeclareCaptionFont{red}{\color{red}}
%% 貼り付けるコードのプログラミング言語
\lstloadlanguages{{[LaTeX]TeX}, {[ISO]C++}, {Java}, {Ruby}, {Python}, {Scala}}
%% listingsパッケージのグローバルスタイル設定
%% 参考:http://hi.baidu.com/shawpinlee/blog/item/9ec431cbae28e41cbe09e6e4.html
\lstset{
numberbychapter=true,
breakatwhitespace=true,
showstringspaces=false,              %% コード間のスペース文字表示の有無
basicstyle=\footnotesize\ttfamily,   %% フォントサイズ設定(\tiny, \scriptsize, \footnotesize, \small, \Largeなど)
keywordstyle=\bfseries,
commentstyle=\color{red!50!green!50!blue!50},                           
escapechar=`,                        %% 日本語エスケープ文字(日英混在用)
xleftmargin=1.5em,xrightmargin=0em, aboveskip=1em,
breaklines,                          %% 長いコード行の自動改行
extendedchars=false,                 %% ページをまたぐコードの章節タイトル、ページヘッダーなどの日本語文字表示問題解決
frameround=fttt,
captionpos=top,
belowcaptionskip=1em
}
\lstdefinestyle{numbers}{
   numbers=left,
   numberstyle=\tiny,
   stepnumber=1,
   numbersep=1em
}
\lstdefinestyle{C++}{
   language=C++,
   texcl=true,
   prebreak=\textbackslash,
   breakindent=1em,
   keywordstyle=\bfseries, %% キーワードハイライト
   morekeywords={alignas, alignof, char16_t, char32_t, constexpr, decltype, noexcept, nullptr, static_assert, thread_local, override, OVERRIDE, INTERFACE, ABSTRACT, DEFINE_ROLE, ROLE, HAS_ROLE, USE_ROLE}
   style=numbers,
   %frame=leftline,                     %% コードに枠を追加
   %framerule=2pt,
   %rulesep=5pt
}
\lstnewenvironment{c++}[1][]
  {\setstretch{1}
  \lstset{style=C++, #1}}
  {}
%\captionsetup[lstlisting]{textfont=red}
%{labelfont=bf, singlelinecheck=off, labelsep=space, textfont=red}
\lstdefinestyle{Java}{
   language=Java,
   texcl=true,
   prebreak=\textbackslash,
   breakindent=1em,
   keywordstyle=\bfseries, %% キーワードハイライト
   morekeywords={}
   style=numbers,
   %frame=leftline,                     %% コードに枠を追加
   %framerule=2pt,
   %rulesep=5pt
}
\lstnewenvironment{java}[1][]
  {\setstretch{1}
  \lstset{style=Java, #1}}
  {}
\lstdefinestyle{Ruby}{
   language=Ruby,
   texcl=true,
   prebreak=\textbackslash,
   breakindent=1em,
   keywordstyle=\bfseries, %% キーワードハイライト
   morekeywords={}
   style=numbers,
   %frame=leftline,                     %% コードに枠を追加
   %framerule=2pt,
   %rulesep=5pt
}
\lstnewenvironment{ruby}[1][]
  {\setstretch{1}
  \lstset{style=Ruby, #1}}
  {}
\lstdefinestyle{Python}{
   language=Python,
   texcl=true,
   prebreak=\textbackslash,
   breakindent=1em,
   keywordstyle=\bfseries, %% キーワードハイライト
   morekeywords={}
   style=numbers,
   %frame=leftline,                     %% コードに枠を追加
   %framerule=2pt,
   %rulesep=5pt
}
\lstnewenvironment{python}[1][]
  {\setstretch{1}
  \lstset{style=Python, #1}}
  {}
\lstdefinestyle{Scala}{
   language=Scala,
   texcl=true,
   prebreak=\textbackslash,
   breakindent=1em,
   keywordstyle=\bfseries, %% キーワードハイライト
   morekeywords={}
   style=numbers,
   %frame=leftline,                     %% コードに枠を追加
   %framerule=2pt,
   %rulesep=5pt
}
\lstnewenvironment{scala}[1][]
  {\setstretch{1}
  \lstset{style=Scala, #1}}
  {}  
\renewcommand{\lstlistingname}{コード例}
\renewcommand\thefigure{\thechapter-\arabic{figure}}
\newcommand\refcode[1]{{\itshape \lstlistingname\ascii{\ref{code:#1}(\pageref{code:#1}ページ)}}}
