%%%% fancyhdrによるヘッダーとフッターの設定
%% ヘッダーとフッターのパッケージ
\usepackage{fancyhdr}
%% ヘッダーとフッターのスタイル
\pagestyle{fancy}
%%これら2行のコードは\leftmarkと\rightmarkの古典的なモードを修正します
\renewcommand{\chaptermark}[1]{\markboth{{\hei {第\thechapter{}章}}\hspace 1  #1}{}}
\renewcommand{\sectionmark}[1]{\markright{\thesection{} #1}}
%% 現在のページのヘッダーとフッターのデフォルト設定をクリア
\fancyhf{}
%\fancyhead[L]{\scriptsize \fangsong \ascii{ZTE}中興}
%\fancyhead[R]{\scriptsize \fangsong 内部公開}
%\fancyhead[CE]{\scriptsize \fangsong \leftmark}
%\fancyhead[CO]{\scriptsize \fangsong \rightmark}
%\fancyfoot[RO, LE]{\scriptsize \thepage}
%\fancyfoot[C]{\scriptsize \fangsong 本文中のすべての情報はZTE Corporation内部情報であり、外部に流出させてはいけません}
\renewcommand{\headrulewidth}{0.4pt}
\renewcommand{\footrulewidth}{0.4pt}
%第{\couriernew\thechapter{}}章
%%以下、ヘッダーとフッターの修正を開始
\fancyhead[RE]{\fangsong \leftmark}
\fancyhead[LO]{\fangsong \rightmark}
\fancyhead[RO, LE]{\small \thepage}
\fancypagestyle{plain}{%
  \fancyhead{} % ヘッダーを削除
  \renewcommand{\headrulewidth}{0pt} % そして線も
}
%%空白ページの定義
\makeatletter
\def\cleardoublepage{\clearpage\if@twoside \ifodd\c@page\else
  \hbox{}
  \vspace*{\fill}
  \begin{center}
   {\sffamily\large}
   \end{center}
   \vspace{\fill}
   \thispagestyle{empty}
   \newpage
   \if@twocolumn\hbox{}\newpage\fi\fi\fi}
\makeatother
\makeatletter
\def\cleardedicatepage{\clearpage
  \hbox{}
  \vspace*{\fill}
  \begin{center}
   {\sffamily\Large 娘の劉楚溪に捧ぐ}
   \end{center}
   \vspace{\fill}
   \thispagestyle{empty}
   \newpage
   \if@twocolumn\hbox{}\newpage\fi}
\makeatother
%% 時々\headheight too smallという警告が出ることがあります
\setlength{\headheight}{15pt}
