%%%%%%%%------------------------------------------------------------------------
%%%% 日常使用マクロパッケージ

%%行間隔の設定
\usepackage{setspace}

%% 箇条書きの制御
\usepackage{enumerate}

%% 複数列表示
\usepackage{multicol}

%% hyperrefパッケージ、位置指定可能なクリック可能なハイパーリンクを生成し、PDFブックマークも生成します
\usepackage[%
    pdfstartview=FitH,%
    CJKbookmarks=true,%
    bookmarks=true,%
    bookmarksnumbered=true,%
    bookmarksopen=true,%
    colorlinks=true,%
    citecolor=blue,%
    linkcolor=blue,%
    anchorcolor=green,%
    urlcolor=blue%
]{hyperref}

%% タイトルの制御
\usepackage{titlesec}

%% テーブルスタイルの制御
\usepackage{booktabs}

%% 目次の制御
\usepackage{titletoc}

%% フォントサイズの制御
\usepackage{type1cm}

%% 段落の最初の行のインデント、\noindentで特定の段落のインデントを解除
\usepackage{indentfirst}

%% カラーテキスト、背景色、テキストボックスなどをサポート
\usepackage{color,xcolor}

%% AMS LaTeXパッケージ
\usepackage{amsmath}
\usepackage{amssymb}

%% 一部の特殊記号
% \usepackage{bbding}

%% 引用のサポート
% \usepackage{cite}

%% LaTeXの一部特殊記号パッケージ
% \usepackage{latexsym}

%% 数式内の太字斜体
% \usepackage{bm}

%% 数式のフォントサイズ調整:\mathsmaller, \mathlarger
% \usepackage{relsize}

%% インデックスの生成
% \makeindex

%%%% 基本的な図の挿入方法
%% グラフィックスパッケージ
\usepackage{graphicx}
\usepackage{float}
%% 挿入する画像の拡張子が指定されていない場合、以下の拡張子を順番に検索します
\DeclareGraphicsExtensions{.pdf,.eps,.png,.jpg}
%% latexに.bbからBounding Box情報を読み取らせる
%\DeclareGraphicsRule{.jpg}{eps}{.bb}{}
%\DeclareGraphicsRule{.png}{eps}{.bb}{}
%\DeclareGraphicsRule{.pdf}{eps}{.bb}{}

%% 複数の図を並べて表示、lnotes.pdfを参照
%\usepackage{subfig}
\usepackage{subfigure}


\usepackage{caption}
\captionsetup{font={sf, scriptsize}, labelfont={bf}, skip=15pt}
\DeclareCaptionLabelSeparator{colon}{~~}

\usepackage[perpage,stable]{footmisc}

\usepackage{longtable}
% \begin{figure}[htbp]               %% 図の位置を制御
%   \setlength{\abovecaptionskip}{0pt}
%   \setlength{\belowcaptionskip}{10pt}
                                     %% 図と前後の文章との距離を制御
%   \centering                       %% 図を中央に配置
%   \includegraphics[width=0.8\textwidth]{CTeXLive2008.jpg}
                                     %% 図の表示幅を0.8\textwidthに制御
%   \caption{CTeXLive2008インストールプロセス} \label{fig:CTeXLive2008}
                                     %% 図のキャプションと相互参照ラベル
% \end{figure}
%%%% 基本的な図の挿入方法終了

%%%% pgf/tikz描画パッケージの設定
\usepackage{pgf,tikz}
\usetikzlibrary{shapes,automata,snakes,backgrounds,arrows}
\usetikzlibrary{mindmap, trees,  calendar}
\usetikzlibrary{positioning}
\usepackage{pgf-umlsd}
%% LaTeXドキュメント内で直接graphviz/dot言語を使用できます
%% または、dot2texツールを使用してdotファイルをtexファイルに変換してincludeすることもできます
%% \usepackage[shell,pgf,outputdir={docgraphs/}]{dot2texi}
%%%% pgf/tikz設定終了


%%%% fancyhdrによるヘッダーとフッターの設定
%% ヘッダーとフッターのパッケージ
\usepackage{fancyhdr}
%% ヘッダーとフッターのスタイル
\pagestyle{fancy}
%%これら2行のコードは\leftmarkと\rightmarkの古典的なモードを修正します
\renewcommand{\chaptermark}[1]{\markboth{{\hei {第\thechapter{}章}}\hspace 1  #1}{}}
\renewcommand{\sectionmark}[1]{\markright{\thesection{} #1}}
%% 現在のページのヘッダーとフッターのデフォルト設定をクリア
\fancyhf{}
%\fancyhead[L]{\scriptsize \fangsong \ascii{ZTE}中興}
%\fancyhead[R]{\scriptsize \fangsong 内部公開}
%\fancyhead[CE]{\scriptsize \fangsong \leftmark}
%\fancyhead[CO]{\scriptsize \fangsong \rightmark}
%\fancyfoot[RO, LE]{\scriptsize \thepage}
%\fancyfoot[C]{\scriptsize \fangsong 本文中のすべての情報はZTE Corporation内部情報であり、外部に流出させてはいけません}
\renewcommand{\headrulewidth}{0.4pt}
\renewcommand{\footrulewidth}{0.4pt}
%第{\couriernew\thechapter{}}章
%%以下、ヘッダーとフッターの修正を開始
\fancyhead[RE]{\fangsong \leftmark}
\fancyhead[LO]{\fangsong \rightmark}
\fancyhead[RO, LE]{\small \thepage}
\fancypagestyle{plain}{%
  \fancyhead{} % ヘッダーを削除
  \renewcommand{\headrulewidth}{0pt} % そして線も
}
%%空白ページの定義
\makeatletter
\def\cleardoublepage{\clearpage\if@twoside \ifodd\c@page\else
  \hbox{}
  \vspace*{\fill}
  \begin{center}
   {\sffamily\large}
   \end{center}
   \vspace{\fill}
   \thispagestyle{empty}
   \newpage
   \if@twocolumn\hbox{}\newpage\fi\fi\fi}
\makeatother
\makeatletter
\def\cleardedicatepage{\clearpage
  \hbox{}
  \vspace*{\fill}
  \begin{center}
   {\sffamily\Large 娘の劉楚溪に捧ぐ}
   \end{center}
   \vspace{\fill}
   \thispagestyle{empty}
   \newpage
   \if@twocolumn\hbox{}\newpage\fi}
\makeatother
%% 時々\headheight too smallという警告が出ることがあります
\setlength{\headheight}{15pt}

%%%% fancyhdr設定終了

%%行間隔の設定
\usepackage{framed}  
%%%% listingsパッケージ設定終了

%%%% 付録設定
\usepackage[title,titletoc,header]{appendix}
%%%% 付録設定終了

%%%% 日常パッケージ設定終了
%%%%%%%%------------------------------------------------------------------------

%%%%%%%%------------------------------------------------------------------------
%%%% 英語フォント設定終了
%% ここに独自の英語フォント設定を追加できます
%%%%%%%%------------------------------------------------------------------------

%%%%%%%%------------------------------------------------------------------------
%%%% よく使用するフォントサイズの設定、MS Wordに対応

%% 一号, 1.4倍行間
\newcommand{\yihao}{\fontsize{26pt}{36pt}\selectfont}
%% 二号, 1.25倍行間
\newcommand{\erhao}{\fontsize{22pt}{28pt}\selectfont}
%% 小二, 単行間
\newcommand{\xiaoer}{\fontsize{18pt}{18pt}\selectfont}
%% 三号, 1.5倍行間
\newcommand{\sanhao}{\fontsize{16pt}{24pt}\selectfont}
%% 小三, 1.5倍行間
\newcommand{\xiaosan}{\fontsize{15pt}{22pt}\selectfont}
%% 四号, 1.5倍行間
\newcommand{\sihao}{\fontsize{14pt}{21pt}\selectfont}
%% 半四, 1.5倍行間
\newcommand{\bansi}{\fontsize{13pt}{19.5pt}\selectfont}
%% 小四, 1.5倍行間
\newcommand{\xiaosi}{\fontsize{12pt}{18pt}\selectfont}
%% 大五, 単行間
\newcommand{\dawu}{\fontsize{11pt}{11pt}\selectfont}
%% 五号, 単行間
\newcommand{\wuhao}{\fontsize{10.5pt}{10.5pt}\selectfont}
%%%%%%%%------------------------------------------------------------------------

%%%%%%%%------------------------------------------------------------------------
%%%% いくつかのカスタム設定

%% ページ番号の方式を設定、arabicやromanなどの方式を含む
%% \pagenumbering{arabic}

%% LaTeXが改行できない場合にoverfullエラーが発生することがあります。このコマンドはLaTeXの改行基準を下げます
%% \sloppy

%% 目次の表示深度を設定\tableofcontents
%% \setcounter{tocdepth}{2}
%% \listoftablesの表示深度を設定
%% \setcounter{lotdepth}{2}
%% \listoffiguresの表示深度を設定
%% \setcounter{lofdepth}{2}

%% 日本語のダッシュ、清華大学のテンプレートから来たとされています
\newcommand{\pozhehao}{\kern0.3ex\rule[0.8ex]{2em}{0.1ex}\kern0.3ex}

%% itemize環境のitemの記号を設定
\renewcommand{\labelitemi}{$\bullet$}

%\makeatletter
%\@addtoreset{lstlisting}{section} 
%\makeatother

\input{style/command}
%% 本文のフォントサイズを設定
% \renewcommand{\normalsize}{\sihao}

%%%% カスタム設定終了
%%%%%%%%------------------------------------------------------------------------


%%%%%%%%------------------------------------------------------------------------
%%%% bibtex設定

%% 参考文献の表示スタイルを設定

%%%% bibtex設定終了
%%%%%%%%------------------------------------------------------------------------
